% Options for packages loaded elsewhere
\PassOptionsToPackage{unicode}{hyperref}
\PassOptionsToPackage{hyphens}{url}
%
\documentclass[
]{article}
\usepackage{amsmath,amssymb}
\usepackage{lmodern}
\usepackage{iftex}
\ifPDFTeX
  \usepackage[T1]{fontenc}
  \usepackage[utf8]{inputenc}
  \usepackage{textcomp} % provide euro and other symbols
\else % if luatex or xetex
  \usepackage{unicode-math}
  \defaultfontfeatures{Scale=MatchLowercase}
  \defaultfontfeatures[\rmfamily]{Ligatures=TeX,Scale=1}
\fi
% Use upquote if available, for straight quotes in verbatim environments
\IfFileExists{upquote.sty}{\usepackage{upquote}}{}
\IfFileExists{microtype.sty}{% use microtype if available
  \usepackage[]{microtype}
  \UseMicrotypeSet[protrusion]{basicmath} % disable protrusion for tt fonts
}{}
\makeatletter
\@ifundefined{KOMAClassName}{% if non-KOMA class
  \IfFileExists{parskip.sty}{%
    \usepackage{parskip}
  }{% else
    \setlength{\parindent}{0pt}
    \setlength{\parskip}{6pt plus 2pt minus 1pt}}
}{% if KOMA class
  \KOMAoptions{parskip=half}}
\makeatother
\usepackage{xcolor}
\usepackage[margin=1in]{geometry}
\usepackage{graphicx}
\makeatletter
\def\maxwidth{\ifdim\Gin@nat@width>\linewidth\linewidth\else\Gin@nat@width\fi}
\def\maxheight{\ifdim\Gin@nat@height>\textheight\textheight\else\Gin@nat@height\fi}
\makeatother
% Scale images if necessary, so that they will not overflow the page
% margins by default, and it is still possible to overwrite the defaults
% using explicit options in \includegraphics[width, height, ...]{}
\setkeys{Gin}{width=\maxwidth,height=\maxheight,keepaspectratio}
% Set default figure placement to htbp
\makeatletter
\def\fps@figure{htbp}
\makeatother
\setlength{\emergencystretch}{3em} % prevent overfull lines
\providecommand{\tightlist}{%
  \setlength{\itemsep}{0pt}\setlength{\parskip}{0pt}}
\setcounter{secnumdepth}{-\maxdimen} % remove section numbering
\ifLuaTeX
  \usepackage{selnolig}  % disable illegal ligatures
\fi
\IfFileExists{bookmark.sty}{\usepackage{bookmark}}{\usepackage{hyperref}}
\IfFileExists{xurl.sty}{\usepackage{xurl}}{} % add URL line breaks if available
\urlstyle{same} % disable monospaced font for URLs
\hypersetup{
  pdftitle={A3: SPL Library Checkouts},
  hidelinks,
  pdfcreator={LaTeX via pandoc}}

\title{A3: SPL Library Checkouts}
\author{}
\date{\vspace{-2.5em}}

\begin{document}
\maketitle

\hypertarget{introduction}{%
\subsubsection{Introduction}\label{introduction}}

An introduction of the data and a description of the trends/books/items
you are choosing to analyze (and why!)

\begin{quote}
\begin{quote}
``The SPL Checkout dataset is a comprehensive record of all books
borrowed from the Seattle Public Library between 2022 and 2023. My goal
is to analyze trends for both physical and digital books and compare
them to identify significant patterns. The dataset includes information
on different types of items, such as ebooks, audiobooks, videos, songs,
and physical books, which I will examine for patterns of borrowing. By
analyzing the trends in the dataset, I hope to identify the most
frequently borrowed items, popular genres, and changing borrowing
patterns over time. Ultimately, this analysis can provide valuable
insights into the reading habits and preferences of Seattle library
patrons, which can inform the library's collection development
strategies and services.
\end{quote}
\end{quote}

\hypertarget{summary-information}{%
\subsubsection{Summary Information}\label{summary-information}}

Write a summary paragraph of findings that includes the 5 values
calculated from your summary information R script

These will likely be calculated using your DPLYR skills, answering
questions such as:

\begin{itemize}
\tightlist
\item
  What is the average number of checkouts for digital items?
\item
  What is the top 10 books with the highest number of checkouts in the
  dataset?
\item
  What is the the total checkouts of physical books by month?
\item
\item
\end{itemize}

Feel free to calculate and report values that you find relevant.

\hypertarget{the-dataset}{%
\subsubsection{The Dataset}\label{the-dataset}}

\begin{itemize}
\item
  Who collected/published the data? \textgreater\textgreater The Seattle
  Public Library SPL
\item
  What are the parameters of the data (dates, number of checkouts, kinds
  of books, etc.)?

  \begin{quote}
  \begin{quote}
  UsageClass CheckoutType MaterialType CheckoutYear CheckoutMonth
  Checkouts Title ISBN Creator Subjects Publisher PublicationYear
  \end{quote}
  \end{quote}
\item
  How was the data collected or generated?\\
  \textgreater Rows 42M Columns 12 Each row is a Checkout Count
\item
  Why was the data collected?\\
  \textgreater\textgreater One reason why someone may have created this
  dataset is to study the patterns of library usage, including which
  types of materials are checked out the most, which authors and
  publishers are preferred, and how usage changes over time. This data
  could help inform decisions about how to develop the library's
  collection, advertise, and allocate resources. Additionally, analyzing
  the data could provide insight into the reading and media consumption
  habits of library visitors. However, it is unclear without more
  information why the dataset was created in the first place.
\item
  What, if any, ethical questions do you need to consider when working
  with this data?
\end{itemize}

\begin{quote}
\begin{quote}
One ethical concern to keep in mind when using the ``Checkouts by Title
SPL'' dataset is the possibility of bias and discrimination. If the data
is used to decide which materials the library should focus on acquiring
or promoting, it could lead to biased decisions that disadvantage
certain authors or genres. For instance, if the data shows that some
types of books are more popular than others, the library might
prioritize them over less popular ones, which could leave out minority
voices and perspectives.
\end{quote}
\end{quote}

\begin{itemize}
\tightlist
\item
  What are possible limitations or problems with this data? (at least
  200 words)
\end{itemize}

\begin{quote}
\begin{quote}
The ``Checkouts by Title SPL'' dataset is useful for investigating
library usage trends, but it also has certain limitations and biases
that need to be taken into account when analyzing the data. One
limitation is that the dataset only covers a specific time period, which
may not allow for long-term trends in library usage to be recognized.
Additionally, data quality is a concern because human errors during data
entry may impact the accuracy and reliability of the dataset.
Furthermore, the dataset's scope is limited to a few variables, such as
title, author, and publication year, which means that it may not be
possible to explore more complex research questions that require
additional variables like patron demographics. Finally, while the
dataset is anonymized, it still includes personal information about
library patrons, which could pose a privacy risk if it were to be
mishandled or accessed by unauthorized individuals.Therefore,
researchers must be cautious when interpreting the findings of this
dataset and draw conclusions carefully. It's important to consider the
dataset's limitations and biases and take steps to address them, such as
accounting for the dataset's limited scope when designing research
questions or checking data quality. By taking these considerations into
account, researchers can use this dataset effectively to identify useful
insights and inform decisions about library collection development,
marketing, and resource allocation.
\end{quote}
\end{quote}

\hypertarget{first-trends-over-time-chart}{%
\subsubsection{First Trends Over Time
Chart}\label{first-trends-over-time-chart}}

Include a chart. Make sure to describe why you included the chart, and
what patterns emerged

The first chart that you will create and include will show the trend
over time of your variable/topic/interest. Think carefully about what
you want to communicate to your user (you may have to find relevant
trends in the dataset first!). Here are some requirements to help guide
your design:

\begin{itemize}
\item
  Show more than one, but fewer than \textasciitilde10 trends

  \begin{itemize}
  \tightlist
  \item
    For example, two books, or 5 different mediums
  \end{itemize}
\item
  You must have clear x and y axis labels
\item
  The chart needs a clear title~
\item
  You need a legend for your different line colors and a clear - legend
  title In your .Rmd file, make sure to describe why you included the
  chart, and what patterns emerged
\end{itemize}

When we say ``clear'' or ``human readable'' titles and labels, that
means that you should not just display the variable name.

\begin{quote}
\begin{quote}
The purpose of this chart was to display the trends over time for the
two most commonly borrowed material types from the Seattle Public
Library, which were physical books and e-books, during the period of
2022 to 2023. The chart illustrated the number of checkouts for both
types of materials over the course of a year, with clear x and y-axis
labels and a legend for different line colors. The chart revealed that
the number of checkouts for physical books remained relatively stable
throughout the year, with a minor increase during the summer months. In
contrast, e-book checkouts steadily increased throughout the year and
peaked in December. This pattern highlighted the rising popularity of
e-books and how they have increasingly become a preferred option for
library patrons over the years.
\end{quote}
\end{quote}

Here's an example of how to run an R script inside an RMarkdown file:

\includegraphics{index_files/figure-latex/unnamed-chunk-1-1.pdf}

\hypertarget{second-trends-over-time-chart}{%
\subsubsection{Second Trends Over Time
Chart}\label{second-trends-over-time-chart}}

Include a chart. Make sure to describe why you included the chart, and
what patterns emerged

The second chart that you will create and include will show another
trend over time of your variable/topic/interest. Think carefully about
what you want to communicate to your user (you may have to find relevant
trends in the dataset first!). Here are some requirements to help guide
your design:

\begin{itemize}
\tightlist
\item
  Show more than one, but fewer than \textasciitilde10 trends

  \begin{itemize}
  \tightlist
  \item
    For example, two books, or 5 different mediums
  \end{itemize}
\item
  You must have clear x and y axis labels
\item
  The chart needs a clear title~
\item
  You need a legend for your different line colors and a clear - legend
  title In your .Rmd file, make sure to describe why you included the
  chart, and what patterns emerged
\end{itemize}

When we say ``clear'' or ``human readable'' titles and labels, that
means that you should not just display the variable name.

Here's an example of how to run an R script inside an RMarkdown file:

\includegraphics{index_files/figure-latex/unnamed-chunk-2-1.pdf}

\hypertarget{your-choice}{%
\subsubsection{Your Choice}\label{your-choice}}

The last chart is up to you. It could be a line plot, scatter plot,
histogram, bar plot, stacked bar plot, and more. Here are some
requirements to help guide your design:

\begin{itemize}
\tightlist
\item
  You must have clear x and y axis labels
\item
  The chart needs a clear title~
\item
  You need a legend for your different line colors and a clear legend
  title
\end{itemize}

Here's an example of how to run an R script inside an RMarkdown file:

\includegraphics{index_files/figure-latex/unnamed-chunk-3-1.pdf}

\end{document}
